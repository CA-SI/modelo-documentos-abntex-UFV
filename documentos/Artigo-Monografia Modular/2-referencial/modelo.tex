\section{Modelo}\label{sec:modelo}
Exemplo de utiliação de imagem em \LaTeX
\begin{figure}[htbp]
    \begin{center}
    \includegraphics[width=.5\linewidth]{assets/LogoUFV}\\
    \end{center}
    \caption[Exemplo de Figura]{Exemplo de inserção de figura no \LaTeX. A legenda deve vir abaixo da figura. Pode usar o comando \texttt{\textbackslash legend} ou \texttt{\textbackslash fonte} para inserir a fonte da figura. Observe que na lista de ilustrações foi utilizado o nome curto fornecido como parâmetro do caption da figura (veja o arquivo fonte .tex) ao invés dessa legenda estupidamente extensa feita de forma proposital}
    \label{fig:logo}
    \legend{Fonte: Próprio Autor}
  \end{figure}

Exemplo de utilização de Código
\begin{minted}{python}
import numpy as np

def incmatrix(genl1,genl2):
    m = len(genl1)
    n = len(genl2)
    M = None #to become the incidence matrix
    VT = np.zeros((n*m,1), int)  #dummy variable
    #compute the bitwise xor matrix
    M1 = bitxormatrix(genl1)
    M2 = np.triu(bitxormatrix(genl2),1) 
    for i in range(m-1):
        for j in range(i+1, m):
            [r,c] = np.where(M2 == M1[i,j])
            for k in range(len(r)):
                VT[(i)*n + r[k]] = 1;
                VT[(i)*n + c[k]] = 1;
                VT[(j)*n + r[k]] = 1;
                VT[(j)*n + c[k]] = 1;
                if M is None:
                    M = np.copy(VT)
                else:
                    M = np.concatenate((M, VT), 1)
                VT = np.zeros((n*m,1), int)
    return M
\end{minted}