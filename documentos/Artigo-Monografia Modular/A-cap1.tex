\chapter{Processo de Desenvolvimento}

\clearpage
\begin{figure}[htbp]
  \begin{center}
  \includegraphics[width=\linewidth]{assets/ProcessoPreDesenvolvimento.jpg}\\
  \end{center}
  \caption[Processo de Pré-Desenvolvimento de Websites e Sistemas para Web]{Processo de Pré-Desenvolvimento de \textit{Websites} e Sistemas para \textit{Web}}
  \label{fig:ProcessoPreDesenvolvimento}
  \legend{Fonte: Próprio Autor}
\end{figure}
\clearpage
\begin{figure}[htbp]
  \begin{center}
  \includegraphics[width=\linewidth]{assets/ProcessoFrontend.jpg}\\
  \end{center}
  \caption[Processo de Desenvolvimento Frontend de Websites e Sistemas para Web]{Processo de Desenvolvimento \textit{Frontend} de \textit{Websites} e Sistemas para \textit{Web}}
  \label{fig:ProcessoDesenvolvimentoFrontend}
  \legend{Fonte: \cite{Roadmap2019} (Adaptado e Traduzido pelo Próprio Autor)}
\end{figure}

\begin{figure}[htbp]
  \begin{center}
  \includegraphics[width=\linewidth]{assets/ProcessoBackend.jpg}\\
  \end{center}
  \caption[Processo de Desenvolvimento Backend de Websites e Sistemas para Web]{Processo de Desenvolvimento \textit{Backend} de \textit{Websites} e Sistemas para \textit{Web}}
  \label{fig:ProcessoDesenvolvimentoBackend}
  \legend{Fonte: \cite{Roadmap2019} (Adaptado e Traduzido pelo Próprio Autor)}
\end{figure}

\begin{figure}[htbp]
  \begin{center}
  \includegraphics[width=\linewidth]{assets/ProcessoDevOps.jpg}\\
  \end{center}
  \caption[Processo de Desenvolvimento DevOps de Websites e Sistemas para Web]{Processo de Desenvolvimento \textit{DevOps} de \textit{Websites} e Sistemas para \textit{Web}}
  \label{fig:ProcessoDevOpsDesenvolvimento}
  \legend{Fonte: \cite{Roadmap2019} (Adaptado e Traduzido pelo Próprio Autor)}
\end{figure}
O processo de pré-desenvolvimento também referido como figura \ref{fig:ProcessoPreDesenvolvimento} engloba áreas como administração, marketing, design visual, gestão de pessoas, tarefas e projetos, engenharia de \textit{software}, arquitetura da informação e interação humano-computador. O processo é subdividido de acordo com o tamanho do projeto para que ações desnecessárias não sejam tomadas, entretanto alguns projetos podem possuir peculiaridades tornando necessário deslocar ações de outra subdivisão. As ações descrita no diagrama acima tem caráter terminativo ou paralelo, sendo as tarefas terminativas necessárias para que seja possível realizar outras tarefas e as em tarefas paralelo com a execução de múltiplas ações simultaneamente.

O processo de desenvolvimento \textit{front-end} exibido na figura \ref{fig:ProcessoDesenvolvimentoFrontend} é constituído de um roteiro de conceitos, métodos e tecnologias necessários para um entendimento global sobre tudo aquilo que permeia o lado do cliente. Por muito tempo o desenvolvimento frontend era basicamente HTML, CSS e um ínfimo e rudimentar JavaScript, com o aperfeiçoamento do JavaScript no quesito de execução de script no navegador e interpretação de código resultou em uma especialização maior da área de frontend. Tornando possível a criação recursos para problemas desconhecidos até então, como por exemplo o \textit{Module Bundler} que é responsável por interconectar e minificar diversas dependências complexas de CSS e JS. Muitas outras ferramentas se apoiaram no avanço do JavaScript e facilitam cada dia mais o desenvolvimento dessa área.

Já o processo de desenvolvimento \textit{back-end} se apoia em conceitos, métodos e tecnologias de diversas áreas do conhecimento, para uma assimilação plena devemos entender áreas como Sistemas Operacionais, Programação de Scripts, aprofundando conhecimento de Segurança da Informação, dentre os demais descritos no diagrama da figura \ref{fig:ProcessoDesenvolvimentoBackend}. Esse processo foi por muito tempo mitigado sem a utilização de boas práticas de desenvolvimento.

O processo de \textit{DevOps} também chamado de figura \ref{fig:ProcessoDevOpsDesenvolvimento} consiste na união de técnicas de desenvolvimento e técnicas de operações. Em sua essência essa etapa constitui em entender o funcionamento de sistemas operacionais para assim poder criar a infraestrutura lógica e física de servidores bem como o monitoramento e controle desses serviços.