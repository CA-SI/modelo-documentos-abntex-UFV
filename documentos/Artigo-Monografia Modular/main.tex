%%==============================================================
%%      Modelo de TCC para o curso de Sistemas de Informação
%%    Universidade Federal de Viçosa - Campus de Rio Paranaíba
%%
%%  Autor: Rodrigo Smarzaro (smarzaro@ufv.br)
%%         Centro Acadêmico de Sistemas de Informação
%%  Última versão Outubro de 2019
%%  Licença: MIT
%%
%%  Codificação UTF-8
%%  Requisitos:
%%              - pdftex
%%              - arquivo abntex2-UFV.sty
%%              - pygments (python pip)
%%==============================================================

% ---
% CONFIGURAÇÕES GLOBAIS
% ---

% Configurações Especiais
\documentclass[
	% -- opções da classe memoir --
	12pt, % tamanho da fonte
	openright, % capítulos começam em pág ímpar (insere página vazia caso preciso)
	oneside, % para impressão só no anverso. Oposto a twoside
	a4paper, % tamanho do papel.
  % -- opções do pacote abntex2 --
  % chapter=TITLE, % Títulos em maiúsculas
  sumario=tradicional, % Sumário padrão memoir (mais bonito "imo")
  % -- opções do pacote babel --
	english, % idioma adicional para hifenização
	brazil, % o último idioma é o principal do documento
]{abntex2} % Personaliza a capa. Precisa do arquivo ufv.cls para funcionar.

% Pacotes Fundamentais
\usepackage{misc/abntex2-UFV} % Personalização para a Universidade Federal de Viçosa
\usepackage{lmodern} % Usa a fonte Latin Modern			
\usepackage[T1]{fontenc} % Seleção de codigos de fonte de saída
\usepackage[utf8]{inputenc} % Codificação do documento (conversão automática dos acentos)
\usepackage{indentfirst} % Indenta o primeiro parágrafo de cada seção
\usepackage{graphicx} % Inclusão de gráficos
\usepackage{booktabs} % \toprule, \midrule e \bottomrule para tabelas
\usepackage{kantlipsum} % Gerador de texto Lorem upslum
\usepackage{minted} % Código em LaTeX
\usepackage[alf,abnt-emphasize=bf]{abntex2cite} % Sistema autor-data com títulos nas referências em negrito
% ---


% ---
% CONFIGURAÇÕES SOBRE O TRABALHO
% ---

% Dados para Capa, Folha de Rosto e Folha de Aprovação
\titulo{Template para Trabalhos de Conclusão de Curso da UFV}
\autor{Rodrigo Smarzaro e CASI}
\local{Rio Paranaíba}
\data{2019}
\orientador{Nome do Orientador}
%\coorientador{Nome do Coorientador}
\instituicao{Universidade Federal de Viçosa}
\campus{\emph{Campus} de Rio Paranaíba}
\curso{Sistemas de Informação}
\membrobancaA{Membro da Banca A}
\membrobancaB[UFMG]{Membro da Banca B}
\databanca{\today}

% Preambulo
% Atenção deve conter o tipo do trabalho, o objetivo, o nome da instituição e a área de concentração
\preambulo{Monografia apresentada à Universidade Federal de Viçosa como parte das exigências para a aprovação na disciplina Trabalho de Conclusão de Curso I}

% Configurações de aparência do PDF final
\makeatletter
% Metadados
\hypersetup{
	pdftitle={\@title},
	pdfauthor={\@author},
    pdfsubject={\imprimirpreambulo},
	 pdfcreator={LaTeX with abnTeX2},
	colorlinks=true, % false: links em frame; true: links coloridos
    linkcolor=black, % cor dos links no documento
    citecolor=blue, % cor dos links para a bibliografia
    filecolor=magenta, % cor dos links para arquivos
	urlcolor=blue, % cor dos links para sites
	bookmarksdepth=4 % profundidade do sumário do PDF
}
\makeatother
% ---

% ----------------------------------------------------------
% ELEMENTOS PRÉ-TEXTUAIS
% ----------------------------------------------------------
\begin{document}
\frenchspacing
\pretextual

% Capa
\imprimircapa

% Folha de Rosto
\imprimirfolhaderosto

% Inserir Folha de Aprovação
%\imprimirfolhadeaprovacao

% Dedicatória
\iffalse % Remova essa linha para descomentar
\begin{dedicatoria}
   \vspace*{\fill}
   \centering
   \noindent
   \textit{Texto qualquer da dedicatória} % Remova essa linha, substitua pela sua dedicatória
   \vspace*{\fill}
\end{dedicatoria}
\fi % Remova essa linha para descomentar

% Agradecimentos
\iffalse % Remova essa linha para descomentar
\begin{agradecimentos}
  \textit{Texto qualquer da dedicatória} % Remova essa linha, substitua pelo seu agradecimento
\end{agradecimentos}
\fi % Remova essa linha para descomentar

% Epígrafe
\iffalse % Remova essa linha para descomentar
\begin{epigrafe}
    \vspace*{\fill}
  	\begin{flushright}
		\textit{``Word? nunca mais.''\\ % Remova essa linha, substitua pela sua epígrafe
		(Qualquer usuário de \LaTeX)} % Remova essa linha, substitua pela sua epígrafe
	\end{flushright}
\end{epigrafe}
\end{agradecimentos}
\fi % Remova essa linha para descomentar

% ---
% Resumos
% ---

% Resumo em Português
\begin{resumo}
  \noindent
  \textit{Texto do seu resumo} % Remova essa linha, substitua pelo seu resumo
  \vspace{\onelineskip}
  \noindent
  \textbf{Palavras-chaves}: Trabalho de Conclusão de Curso, abntex, LaTeX, UFV.
\end{resumo}

% Resumo em Inglês
\begin{resumo}[Abstract]
  \begin{otherlanguage*}{english}
    \noindent
    \textit{Abstract text} % Remova essa linha, substitua pelo seu abstract
    \vspace{\onelineskip}
    \noindent
    \textbf{Key-words}: Term Paper, abntex, LaTeX, UFV.
  \end{otherlanguage*}
\end{resumo}
% ---


% Lista de Ilustrações
\pdfbookmark[0]{\listfigurename}{lof}
\listoffigures*
\cleardoublepage

% Lista de Tabelas
\pdfbookmark[0]{\listtablename}{lot}
\listoftables*
\cleardoublepage

% Lista de Siglas e Abreviaturas (opcional)
% Sintaxe: \item [sigla] Descrição da sigla
\iffalse % Remova essa linha para descomentar
\begin{siglas}
  \item[ABNT] Absurdas Normas Técnicas
  \item[UFV] Universidade Federal de Viçosa
  \item[CRP] \emph{Campus} de Rio Paranaíba
\end{siglas}
\fi % Remova essa linha para descomentar

% Lista de símbolos (opcional)
% sintaxe: \item [simbolo] Descrição do símbolo
\iffalse % Remova essa linha para descomentar
\begin{simbolos}
  \item[$\infty$] Infinito
\end{simbolos}
\fi % Remova essa linha para descomentar

% Sumario
\pdfbookmark[0]{\contentsname}{toc}
\tableofcontents*
\cleardoublepage
% ---


% ----------------------------------------------------------
% ELEMENTOS TEXTUAIS
% ----------------------------------------------------------

\textual

\chapter{Introdução}\label{sec:introducao}
Alguns links interessantes para se trabalhar com a classe abn\TeX\ e \LaTeX\ em geral\footnote{E também para usar alguns comandos de citação como exemplo}:
\begin{alineas}
  \item Informações da classe Abn\TeX : \citeonline{abntex2classe}
  \item Ajustes nas citações e referências: \citeonline{abntex2cite} e \citeonline{abntex2cite-alf}
  \item Classe memoir (base do Abn\TeX\ ): \apudonline{memoir}{abntex2classe}
  \item Livros interessantes sobre \LaTeX: \cite{Dongen2012,LeslieLamport90,FrankMittelbach111,Dongen2012}
  \item Distribuição \LaTeX\ para windows: \url{http://miktex.org/}
  \item Gerenciador de arquivos \texttt{.bib}: \url{http://jabref.sourceforge.net/}
  \item Gerenciador de artigos: \url{http://www.mendeley.com/}
  \item Exemplo de Tabela: Veja \autoref{tab:cronograma}
\end{alineas}
\section{Motivação}\label{sec:motivacao}

\section{Objetivos}\label{sec:objetivos}
\subsection{Objetivo Geral}\label{sec:objetivo-geral}


\subsection{Objetivos Específicos}\label{sec:objetivos-especificos}
\begin{itemize}
    \item 
\end{itemize}
\section{Estrutura}\label{sec:estrutura}
\chapter{Referencial Teórico}\label{sec:ref-teorico}
\section{Modelo}\label{sec:modelo}
Exemplo de utiliação de imagem em \LaTeX
\begin{figure}[htbp]
    \begin{center}
    \includegraphics[width=.5\linewidth]{assets/LogoUFV}\\
    \end{center}
    \caption[Exemplo de Figura]{Exemplo de inserção de figura no \LaTeX. A legenda deve vir abaixo da figura. Pode usar o comando \texttt{\textbackslash legend} ou \texttt{\textbackslash fonte} para inserir a fonte da figura. Observe que na lista de ilustrações foi utilizado o nome curto fornecido como parâmetro do caption da figura (veja o arquivo fonte .tex) ao invés dessa legenda estupidamente extensa feita de forma proposital}
    \label{fig:logo}
    \legend{Fonte: Próprio Autor}
  \end{figure}

Exemplo de utilização de Código
\begin{minted}{python}
import numpy as np

def incmatrix(genl1,genl2):
    m = len(genl1)
    n = len(genl2)
    M = None #to become the incidence matrix
    VT = np.zeros((n*m,1), int)  #dummy variable
    #compute the bitwise xor matrix
    M1 = bitxormatrix(genl1)
    M2 = np.triu(bitxormatrix(genl2),1) 
    for i in range(m-1):
        for j in range(i+1, m):
            [r,c] = np.where(M2 == M1[i,j])
            for k in range(len(r)):
                VT[(i)*n + r[k]] = 1;
                VT[(i)*n + c[k]] = 1;
                VT[(j)*n + r[k]] = 1;
                VT[(j)*n + c[k]] = 1;
                if M is None:
                    M = np.copy(VT)
                else:
                    M = np.concatenate((M, VT), 1)
                VT = np.zeros((n*m,1), int)
    return M
\end{minted}
\chapter{Trabalhos Relacionados}\label{sec:trab-rel}

\chapter{Metodologia}\label{sec:metodos}
\clearpage
\input{4-metodologia/cronograma.tex}
% Adicione ou tire capítulos com o seu gosto
\chapter{Processo de Desenvolvimento}

\clearpage
\begin{figure}[htbp]
  \begin{center}
  \includegraphics[width=\linewidth]{assets/ProcessoPreDesenvolvimento.jpg}\\
  \end{center}
  \caption[Processo de Pré-Desenvolvimento de Websites e Sistemas para Web]{Processo de Pré-Desenvolvimento de \textit{Websites} e Sistemas para \textit{Web}}
  \label{fig:ProcessoPreDesenvolvimento}
  \legend{Fonte: Próprio Autor}
\end{figure}
\clearpage
\begin{figure}[htbp]
  \begin{center}
  \includegraphics[width=\linewidth]{assets/ProcessoFrontend.jpg}\\
  \end{center}
  \caption[Processo de Desenvolvimento Frontend de Websites e Sistemas para Web]{Processo de Desenvolvimento \textit{Frontend} de \textit{Websites} e Sistemas para \textit{Web}}
  \label{fig:ProcessoDesenvolvimentoFrontend}
  \legend{Fonte: \cite{Roadmap2019} (Adaptado e Traduzido pelo Próprio Autor)}
\end{figure}

\begin{figure}[htbp]
  \begin{center}
  \includegraphics[width=\linewidth]{assets/ProcessoBackend.jpg}\\
  \end{center}
  \caption[Processo de Desenvolvimento Backend de Websites e Sistemas para Web]{Processo de Desenvolvimento \textit{Backend} de \textit{Websites} e Sistemas para \textit{Web}}
  \label{fig:ProcessoDesenvolvimentoBackend}
  \legend{Fonte: \cite{Roadmap2019} (Adaptado e Traduzido pelo Próprio Autor)}
\end{figure}

\begin{figure}[htbp]
  \begin{center}
  \includegraphics[width=\linewidth]{assets/ProcessoDevOps.jpg}\\
  \end{center}
  \caption[Processo de Desenvolvimento DevOps de Websites e Sistemas para Web]{Processo de Desenvolvimento \textit{DevOps} de \textit{Websites} e Sistemas para \textit{Web}}
  \label{fig:ProcessoDevOpsDesenvolvimento}
  \legend{Fonte: \cite{Roadmap2019} (Adaptado e Traduzido pelo Próprio Autor)}
\end{figure}
O processo de pré-desenvolvimento também referido como figura \ref{fig:ProcessoPreDesenvolvimento} engloba áreas como administração, marketing, design visual, gestão de pessoas, tarefas e projetos, engenharia de \textit{software}, arquitetura da informação e interação humano-computador. O processo é subdividido de acordo com o tamanho do projeto para que ações desnecessárias não sejam tomadas, entretanto alguns projetos podem possuir peculiaridades tornando necessário deslocar ações de outra subdivisão. As ações descrita no diagrama acima tem caráter terminativo ou paralelo, sendo as tarefas terminativas necessárias para que seja possível realizar outras tarefas e as em tarefas paralelo com a execução de múltiplas ações simultaneamente.

O processo de desenvolvimento \textit{front-end} exibido na figura \ref{fig:ProcessoDesenvolvimentoFrontend} é constituído de um roteiro de conceitos, métodos e tecnologias necessários para um entendimento global sobre tudo aquilo que permeia o lado do cliente. Por muito tempo o desenvolvimento frontend era basicamente HTML, CSS e um ínfimo e rudimentar JavaScript, com o aperfeiçoamento do JavaScript no quesito de execução de script no navegador e interpretação de código resultou em uma especialização maior da área de frontend. Tornando possível a criação recursos para problemas desconhecidos até então, como por exemplo o \textit{Module Bundler} que é responsável por interconectar e minificar diversas dependências complexas de CSS e JS. Muitas outras ferramentas se apoiaram no avanço do JavaScript e facilitam cada dia mais o desenvolvimento dessa área.

Já o processo de desenvolvimento \textit{back-end} se apoia em conceitos, métodos e tecnologias de diversas áreas do conhecimento, para uma assimilação plena devemos entender áreas como Sistemas Operacionais, Programação de Scripts, aprofundando conhecimento de Segurança da Informação, dentre os demais descritos no diagrama da figura \ref{fig:ProcessoDesenvolvimentoBackend}. Esse processo foi por muito tempo mitigado sem a utilização de boas práticas de desenvolvimento.

O processo de \textit{DevOps} também chamado de figura \ref{fig:ProcessoDevOpsDesenvolvimento} consiste na união de técnicas de desenvolvimento e técnicas de operações. Em sua essência essa etapa constitui em entender o funcionamento de sistemas operacionais para assim poder criar a infraestrutura lógica e física de servidores bem como o monitoramento e controle desses serviços.
\chapter{Resultados}\label{sec:resultados}
\section{Resultados Esperados}
O presente trabalho pretende em etapas posteriores aplicar o método proposto em empresas reais, mensurando antes e depois da implementação a fim de tornar válido e provar a efetividade de tal. Os indicadores associados as métricas serão de produtividade, efetividade, desempenho e indicadores operacionais. Os mesmos evidenciarão se o processo e as procedimentações propostas são definitivamente úteis e conseguem por si só, guiar e amparar profissionais e empresas que trabalham com desenvolvimento para internet. Toda proposta do método realiza escolhas da pilha de solução que deve ser melhor avaliada afim de concluir se a escolha é simplificada e satisfatória.
\chapter{Conclusão e Trabalhos Futuros}
\section{Conclusão}
Percebe-se que a linguagem de programação que o gerador de site estático é implementado influencia diretamente no tempo de geração das páginas.

\section{Trabalhos Futuros}
Para que seja possível dar continuidade nesse trabalho é necessário implementar a ferramenta dstat que monitora recursos de hardware em sistemas operacionais unix para não somente avaliar o tempo de criação das páginas estáticas com os diferentes geradores de sites estáticos mas também os recursos de processador, memoria e disco gastos. Após está rodada que pode-se ser chamada de rodada 0 (zero), a implementação será capaz de executar o dstat na maquina cliente e armazenar em um arquivo de texto

% ---


% ----------------------------------------------------------
% ELEMENTOS PÓS-TEXTUAIS
% ----------------------------------------------------------

\postextual

% Referências Bibliográficas
\bibliography{referencias}

% Caso sejam necessários apêndices ou anexos em seu documento, use os ambientes abaixo

%% Apêndices
\iffalse % Remova essa linha para descomentar
\begin{apendicesenv}
  \partapendices
  \chapter{Primeiro Apêndice}
  \lipsum[1][1]
  \chapter{Segundo Apêndice}
  \lipsum[1][1]
\end{apendicesenv}
\fi % Remova essa linha para descomentar

%% Anexos
%\begin{anexosenv}
\iffalse % Remova essa linha para descomentar
  \partanexos
  \chapter{Primeiro Anexo}
  \lipsum[1][1]
  \chapter{Segundo Anexo}
  \lipsum[1][1]
\end{anexosenv}
\fi % Remova essa linha para descomentar

\end{document}
